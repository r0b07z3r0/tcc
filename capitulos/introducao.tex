%INTRODUCAO - TCC-2 ALEX SOARES DUARTE
\chapter{Introdu\c{c}\~ao}
%Focar em uma coisa mais geral


%Falar sobre automacao industrial; sistemas de evento discreto; dificuldade de projeto
%Falar que tem um grande problema


%contextualizacao-1
Este trabalho propro\~oe uma ferramenta digital para a convers\~ao de sistemas modelados em Redes de Petri para códigos implementáveis em controladores industriais.
Estamos cercados de sistemas a eventos discretos (SEDs). Um sistema a eventos discretos \'e definido como um sistema cuja evolu\c{c}\~ao din\^amica depende da ocorr\^encia de eventos, os quais produzem as mudan\c{c}as de estado e, de modo geral, ocorrem em instantes de tempo irregulares \cite{Montgomery2004}. 

Computadores, redes de comunica\c{c}\~oes, sistemas industriais automatizados, controle de tr\'afego a\'ereo s\~ao alguns exemplos de SEDs. Esses sistemas s\~ao modelados para que assim os usu\'arios possam, com o modelo, entender melhor o comportamento do sistema atrav\'es de an\'alises que os formalismos matem\'aticos proporcionam ao usu\'ario \cite{Wolfgang2013}.

%SEDs s\~ao sistemas modelados de tal sorte que os valores das vari\'aveis nos estados seguintes podem ser calculados diretamente a partir dos valores %precedentes sem ter que considerar o tempo entre estes dois instantes \cite{Janette}.

V\'arios formalismos matem\'aticos podem ser considerados na modelagem de SEDs; Redes de Petri (RdP), Cadeias de Markov, teoria de linguagens e aut\^omatos (m\'aquinas de estados finitos) s\~ao alguns exemplos. Entretanto, n\~ao h\'a um formalismo universal que solucione todos os problemas referentes aos SEDs \cite{Montgomery2004}.

%Um aut\^omato pode ser representado graficamente como o um grafo dirigido, onde os n\'os representam os estados e os arcos etiquetados representam as transi\c{c}\~oes entre os estados \cite{apostilacury}.

Em Redes de Petri, os eventos s\~ao manipulados de acordo com certas regras semelhante aos aut\^omatos. Ela pode ser descrita graficamente e \'e composto por elementos estruturais; lugares, transi\c{c}\~oes, arcos e fichas \cite{Cassandras2008}. Seu grande diferencial fica por conta da an\'alise de suas propriedades, tais como: limitabilidade, quando o n\'umero de fichas em um lugar n\~ao exceda um n\'umero finito, alcan\c{c}abilidade, quando um dado estado \'e alcan\c{c}\'avel a partir de uma sequ\^encia de transi\c{c}\~oes, vivacidade, quando sempre existir ao menos uma transic{c}\~ao habilitada para disparo e evitando um estado de bloqueio, entre outras propriedades \cite{Cassandras2008}.

%O conjunto de marca\c{c}\~oes acess\'iveis de uma Rede de Petri marcada \'e o conjunto das marca\c{c}\~oes que podem ser atingidas a partir da marca\c{c}\~ao inicial atrav\'es de uma sequ\^encia de disparos, este pode ser representado atrav\'es de um aut\^omato (grafo de alcan\c{c}abilidade) que representa a Rede de Petri \cite{Janette}.

Segundo \cite{Montgomery2004}, um problema de controle SEDs pode ser resolvido por meio da Teoria de Controle Supervis\'orio (TCS). A formula\c{c}\~ao de um problema de controle de SEDs \'e definida em tr\^es etapas: modelagem, especifica\c{c}\~ao de comportamento e s\'intese do supervisor. Modelagem \'e a etapa na qual se utiliza um formalismo para representar o SED, e que permite determinar seus estados e sua evolu\c{c}\~ao din\^amica. Especifica\c{c}\~ao de comportamento expressa, atrav\'es de um modelo formal, as tarefas que o sistema deve realizar para resultar em um comportamento desejado. A s\'intese do supervisor \'e uma l\'ogica de controle que soluciona o problema de controle, esse supervisor observa os eventos e define uma sequ\^encia de a\c{c}\~oes de controle que garante um comportamento especificado.

O acoplamento do supervisor gerado pela TCS e a RdP \'e foi introduzida por \cite{UzamWonham2005}. Est\a metodologia \'e o ponto focal deste projeto, pois ela permite computar um modelo de RdP n\~ao bloqueante e limitado dado uma RdP n\~ao limitada e um conjunto de especifica\c{c}\~oes \cite{UzamWonham2005}


%Assuming  that  an  uncontrolled  bounded  Petri  net  (PN)model  of  a (plant)  DES and  a set of forbidden  state specifica-tions  are  given,  the  proposed  approach  computes  a  maximallypermissive and nonblocking closed-loop hybrid model.



%Este trabalho visa contribuir para que a implanta\c{c}\~ao de controladores baseados num modelo formal seja empregada para o controle de SEDs, partindo da l\'ogica sintetizada pela TCS, atrav\'es de uma interface para desenvolvimento de projetos de automa\c{c}\~ao que coverter\'a a modelagem em Redes de Petri em c\'odigos implement\'aveis para controladores l\'ogicos program\'aveis (CLP) ou FPGAs (\textit{Field Programmagle Gate Array}).

O trabalho est\'a organizado em seis cap\'itulos. Nesta introdu\c{c}~ao, se discute sobre a motiva\c{c}~ao e objetivos do projeo. O segundo e terceiro cap\'itulo se refere a fundamenta\c{c}~ao te\'orica de SEDs e Controladores Industriais. Em seguida, no quarto cap\'itulo, uma apresenta\{c}~ao do algoritmo de convers\c{c}~ao e sua interface gr\'afica. E por fim, nos cap\'itulos cinco e seis, a valida\c{c}~ao da ferramenta e a conclus\~ao dos resultados obtidos.

%Este trabalho propo\~oe uma ferramenta para atuar como interface de desenvolvimento de projeto de controle de SEDs. Esta interface tem por objetivo auxiliar o uso da modelagem formal de um sistema convertendo o seu conte\'udo em c\'odigos implement\'aveis. Isto proporcionaria grande redu\c{c}\~ao de tempo de projeto, aumento de seguran\c{c}a e confiabilidade.

%A proposta vem para contribuir com o desenvolvimento que se tem feito cerca da implanta\c{c}\~ao da modelagem de sistemas discretos em c\'odigos para equipamentos de controle e automa\c{c}\~ao program\'aveis. A ideia do trabalho \'e disponibilizar uma ferramenta computacional para preencher brecha entre teoria e aplica\c{c}\~ao.


% qual a partir do documento do gráfico de alcançabilidade gerado pelo software TINA seja computado para um formato adequado para UMDEES 

%-contextualização
%-descrição da situação do documento
%-inicia por que comecou até chegar ao meu trabalho

%Justificativa

\chapter{Justificativa}

%Qual 'e a vantagem de ter isso
%Aproveitamento da teoria para projetos no campo
%Significa mais eficiencia e reducao de tempo
%Maior qualidade de controle e confiabilidade
%Diminuir riscos

A Teoria de Controle Supervis\'orio \'e uma ferramenta apropriada para a s\'intese da l\'ogica de controle para sistemas automatizados \cite{leal2009}. Os mesmos autores confirmam que a TCS garante o alcance de uma l\'ogica de controle \'otima, isto \'e, uma l\'ogica minimamente restritiva e sem bloqueios que atenda as especifica\c{c}\~oes de controle.

As Redes de Petri s\~ao adequadas para a modelagem de controladores de sistemas de automa\c{c}\~ao \cite{gomes2007}. A RdP possui vizualiza\c{c}\~ao gr\'afica, o que permite uma sintaxe e sem\^antica bem definida, como tamb\'em a modelagem concorrente e s\'incrona expl\'icita e leg\'ivel.

Segundo \cite{Litz2000} h\'a v\'arias raz\~oes para a aplica\c{c}\~ao de m\'etodos formais. As principais raz\~oes s\~ao: aumento da complexidade de problemas, demanda por redu\c{c}\~ao do tempo de desenvolvimento, demanda por maior qualidade e demanda por confiabilidade dos sistemas.

Em rela\c{c}\~ao \`a gera\c{c}\~ao autom\'atica de c\'odigos, o relat\'orio do "\textit{Workshop on Logic Control for Manufacturing Systems}" \space aponta que os t\'opicos para pesquisas futuras incluem o aprimoramento de programas para projetos de l\'ogica de controle e implementação atrav\'es de diagn\'osticos integrados. Como tamb\'em melhoria em inferfaces homem-m\'aquina e gera\c{c}\~ao autom\'atica de c\'odigos \cite{workshop2000}. \cite{karenides} cita que estudantes que desenvolvem pesquisa em SEDs desejam que um software de modelagem contenha um gerador e executor autom\'atico de c\'odigos, e que este proporcione uma interface entre software, hardware e a implementa\c{c}\~ao do controlador.

 Este trabalho visa preencher a lacuna entre teoria e implementa\c{c}\~ao com uma ferramenta computacional para gerar c\'odigos para sistemas baseados em CLPs e FPGAs. Atrav\'es de uma interface intuitiva ser\'a poss\'ivel modelar controladores de forma eficiente, utilizando os formalismos para garantir solu\c{c}\~ao \'otima em curto prazo de tempo com a gera\c{c}\~ao autom\'atica de c\'odigos implement\'aveis. O desenvolvimento de uma interface com as caracter\'isticas anunciadas anteriormente resultaria em um aumento da utiliza\c{c}\~ao de t\'ecnicas formais para a s\'intese de controle, o que levaria ao aumento da qualidade dos sistemas de automa\c{c}\~ao, com redu\c{c}\~ao de falhas e aumento da confiabilidade em seguran\c{c}a e desempenho.
 
 %resulta em uma pr\'atica industrial veloz e segura.

%A gera\c{c}\~ao autom\'atica de c\'odigo aumentaria a utiliza\c{c}\~ao de t\'ecnicas formais de s\'intese de controle. O uso de t\'ecnicas formais levaria ao aumento da qualidade dos sistemas de automa\c{c}\~ao, com redu\c{c}\~ao de falhas e aumento de confiabilidade e seguran\c{c}a.


%Em um controlador baseado em Redes de Petri que sintet\'iza a teoria de controle supervis\'orio, o primeiro passo \'e a modelagem da planta e da especifica\c{c}\~ao e depois \'e feita a composi\c{c}\~ao s\'incrona entre os modelos para obter o modelo do control\'avel e h\'a v\'arias maneiras de resolver esse problema \cite{dideban2011}. O diferencial deste trabalho est\'a na possibilidade de sintetizar este controlador de forma autom\'atica seguindo certas regras abordadas pela metodologia.


%In PN-based controller synthesizes using SCT framework, the first  step  is  the  modelling  of  the  plant  and  the  specifications  and then in the next step, synchronized composition between two  models  gets  the  controlled  model.  Generally,  due  to  uncontrollable and unobservable transitions, it is necessary to change this synchronized model. There are many approaches for  resolving  this  problem.   The   final   model   is   composed   of   the supervisor and the uncontrolled model of the plant. The next step is implementing this controller.

%automatic  code  generation  and  code  execution;  this would  allow  hardware to  be  interfaced  with  the  soft-ware,  software  verification,  and  controller  implemen-tation;

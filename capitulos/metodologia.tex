%metodologia

\chapter{Metodologia}

%\section{Introdu\c{c}\~ao}

%Neste cap\'ipulo \'e discutido sobre a metodologia para a proposta de trabalho.

%\section{Metodologia}

%Colocar sobre o estudo de rdp e tcs

Para a resolu\c{c}\~ao do problema proposto e atingimento dos objetivos, \'e proposta uma metodologia de cinco passos.




\section{Etapas de trabalho}

\begin{itemize}
	\item Etapa 1: Revis\~ao bibliogr\'afica;
	\item Etapa 2: Desenvolvimento de algoritmos de convers\~ao e gera\c{c}\~ao;
	\item Etapa 3: Desenvolvimento de um programa computacional;
	\item Etapa 4: An\'alise dos resultados;
	%\item Compara\c{c}\~ao com outras metodologias;
	\item Etapa 5: Escrita da monografia;
	\item Etapa 6: Defesa da monografia.
\end{itemize}

\subsection{Revis\~ao bibliogr\'afica}

Esta etapa consiste em estudar a bibliografia desenvolvida na \'area de SEDs e linguagens de programa\c{c}\~ao de CLPs e FPGAs.

\subsection{Desenvolvimento de algoritmos de convers\~ao e gera\c{c}\~ao} 

Os softwares a serem utilizados n\~ao s\~ao compat\'iveis entre si, assim surge a necessidade de um conversor para que as informa\c{c}\~oes contidas nos arquivos possam ser lidas e modificados por diferentes plataformas. Um gerador de c\'odigos implement\'aveis ser\'a desenvolvido, este gerador ser\'a baseado a partir da combina\c{c}\~ao das caracter\'isticas entre os softwares de modelagem e a linguagem de programa\c{c}\~ao dos controladores, assim um arquivo que contenha a s\'intese da l\'ogica de controle ser\'a gerado e estar\'a pronto para ser implementado em CLPs ou FPGAs.
%Os arquivos do software TINA possuem caracter\'isticas diferentes dos arqu aceitos pelo UMDES, sendo assim necess\'ario a elabora\c{c}\~ao de algoritmos para a convers\~ao destes arquivos. A gera\c{c}\~ao do c\'odigo implement\'avel tamb\'em requer um algoritmo para que se possa adequar o arquivo que representa uma m\'aquina de estados finitos do controlador gerado pelo UMDES para um arquivo no modelo do GRAFCET que seja aceit\'avel pelo CLP.

\subsection{Desenvolvimento de um programa computacional}

\'E necess\'ario o desenvolvimento de um programa que implemente os algoritmos discutidos e que tamb\'em apresente uma interface para auxiliar o usu\'ario no desenvolvimento de projetos de automa\c{c}\~ao.

\subsection{An\'alise dos resultados}

Implantar a proposta em um problema conhecido e analisar os resultados.

%\subsection{Compara\c{c}\~ao com outras metodologias}

%Nesta etapa \'e prevista a execu\c{c}\~ao de outras metodologias de convers\~ao e comparar seus resultados com os resultados obtidos na etapa anterior.

\subsection{Escrita da monografia}

Elabora\c{c}\~ao de um documento para a avalia\c{c}\~ao da banca como parte dos requisitos.

\subsection{Defesa da monografia}

Defesa do trabalho de conclus\~ao de curso perante a banca examinadora.
%\section{Conclus\~ao}

%Neste cap\'itulo foi apresentado a metodologia que compo\~e a proposta de uma ferramenta capaz de gerar c\'odigos implemt\'aveis, onde esta obdece \`as regras previamente discutidas e as etapas para o trabalho de conclus\~ao de curso.
